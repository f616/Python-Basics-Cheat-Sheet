%-----------------------------------------------------
\begin{alerttextbox}{Introductory Note}
This document is an adaption of the original datacamp.org cheat sheet.\\
\begin{itemize}
    \item {https://www.datacamp.com/resources/cheat-sheets/python-for-data-science-cheat-sheet-for-beginners}
    \item {https://github.com/f616/Python-Basics-Cheat-Sheet}
\end{itemize}

\end{alerttextbox}


%-----------------------------------------------------
\section{Variables and Data Types}

\begin{codebox}{python}{Variable Assignment}
x=5
x
>>> 5
\end{codebox}

\begin{codebox}{python}{Calculations With Variables}
x+2  #Sum of two variables
>>> 7

x-2  #Subtraction of two variables
>>> 3

x*2  #Multiplication of two variables
>>> 10

x**2  #Exponentiation of a variable
>>> 25

x%2  #Remainder of a variable
>>> 1

x/float(2)  #Division of a variable
>>> 2.5
\end{codebox}

\begin{codebox}{python}{Types and Type Conversion}
str()
'5', '3.45', 'True'  #Variables to strings

int()
5, 3, 1  #Variables to integers

float()
5.0, 1.0  #Variables to floats

bool()
True, True, True  #Variables to booleans
\end{codebox}


%--------------------------------------------------------------
\section{Libraries}

\begin{myblock}{}
\textbf{pandas:} Data analysis\\
\textbf{NumPy:} Scientific computing\\
\textbf{matplotlib:} 2D plotting\\
\textbf{scikit-learn:} Machine learning\\

\end{myblock}

\begin{codebox}{python}{Import Libraries}
import numpy
import numpy as np
\end{codebox}

\begin{codebox}{python}{Selective import}
from math import pi
\end{codebox}


%--------------------------------------------------------------
\section{Strings}

\begin{myblock}{}
\begin{codebox}{python}{}
my_string = 'thisStringIsAwesome'
my_string
>>> 'thisStringIsAwesome'
\end{codebox}

\begin{codebox}{python}{String Operations}
my_string * 2
>>> 'thisStringIsAwesomethisStringIsAwesome'

my_string + 'Innit'
>>> 'thisStringIsAwesomeInnit'

'm' in my_string
>>> True
\end{codebox}

\green{\emph{Index starts at 0}}
\begin{codebox}{python}{String Indexing}
my_string[3]
my_string[4:9]
\end{codebox}

\begin{codebox}{python}{String Methods}
my_string.upper()  #String to uppercase
my_string.lower()  #String to lowercase
my_string.count('w')  #Count String elements
my_string.replace('e', 'i') #Replace String elements
y_string.strip()  #Strip whitespaces
\end{codebox}
\end{myblock}

%--------------------------------------------------------------
\section{NumPy Arrays}

\begin{myblock}{}
\begin{textbox}{}
\textbf{Also see Lists}
\end{textbox}

\begin{codebox}{python}{}
my_list = [1, 2, 3, 4]
my_array = np.array(my_list)
my_2darray = np.array([[1,2,3],[4,5,6]])
\end{codebox}

\begin{myblock}{Selecting Numpy Array Elements}
\green{\emph{Index starts at 0}}\\
\textbf{Subset}
\begin{codebox}{python}{}
my_array[1]  #Select item at index 1
>>> 2
\end{codebox}

\textbf{Slice}
\begin{codebox}{python}{}
my_array[0:2]  #Select items at index 0 and 1
>>> array([1, 2])
\end{codebox}

\textbf{Subset 2D Numpy arrays}
\begin{codebox}{python}{}
my_2darray[:,0]  #my_2darray[rows, columns]
>>> array([1, 4])
\end{codebox}
\end{myblock}

\begin{codebox}{python}{Numpy Array Operations}
my_array > 3
>>> array([False ,False ,False ,True], dtype=bool)

my_array * 2
>>> array([2, 4, 6, 8])

my_array + np.array([5, 6, 7, 8])
>>> array([6, 8, 10, 12])
\end{codebox}

\begin{codebox}{python}{Numpy Array Functions}
my_array.shape  #Get the dimensions of the array
np.append(other_array)  #Append items to an array
np.insert(my_array, 1, 5)  #Insert items in an array
np.delete(my_array,[1])  #Delete items in an array
np.mean(my_array)  #Mean of the array
np.median(my_array)  #Median of the array
my_array.corrcoef()  #Correlation coefficient
np.std(my_array)  #Standard deviation
\end{codebox}

\end{myblock}

%--------------------------------------------------------------
\section{Lists}

\begin{textbox}{}
\textbf{Also see NumPy Arrays}
\end{textbox}

\begin{codebox}{python}{}
a = 'is'
b = 'nice'
my_list = ['my', 'list', a, b]
my_list2 = [[4,5,6,7], [3,4,5,6]]
\end{codebox}

\begin{myblock}{Selecting List Elements}
\green{\emph{Index starts at 0}}\\
\textbf{Subset}
\begin{codebox}{python}{}
my_list[1]  #Select item at index 1
my_list[-3]  #Select 3rd last item
\end{codebox}

\textbf{Slice}
\begin{codebox}{python}{}
my_list[1:3]  #Select items at index 1 and 2
my_list[1:]  #Select items after index 0
my_list[:3]  #Select items before index 3
my_list[:]  #Copy my_list
\end{codebox}

\textbf{Subset Lists of Lists}
\begin{codebox}{python}{}
my_list2[1][0]  #my_list[list][itemOfList]
my_list2[1][:2]
\end{codebox}

\end{myblock}

\begin{codebox}{python}{List Operations}
my_list + my_list
>>> ['my', 'list', 'is', 'nice', 'my', 'list', 'is', 'nice']

my_list * 2
>>> ['my', 'list', 'is', 'nice', 'my', 'list', 'is', 'nice']

my_list2 > 4
>>> True
\end{codebox}

\begin{codebox}{python}{List Methods}
my_list.index(a)  #Get the index of an item
my_list.count(a)  #Count an item
my_list.append('!')  #Append an item at a time
my_list.remove('!')  #Remove an item
del(my_list[0:1])  #Remove an item
my_list.reverse()  #Reverse the list
my_list.extend('!')  #Append an item
my_list.pop(-1)  #Remove an item
my_list.insert(0,'!')  #Insert an item
my_list.sort()  #Sort the list
\end{codebox}


%--------------------------------------------------------------
\section{Asking For Help}

\begin{codebox}{python}{}
help(str)
\end{codebox}
